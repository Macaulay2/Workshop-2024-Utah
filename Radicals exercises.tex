\documentclass[11pt, oneside]{article}   	% use "amsart" instead of "article" for AMSLaTeX format
\usepackage{geometry}                		% See geometry.pdf to learn the layout options. There are lots.
\geometry{letterpaper}                   		% ... or a4paper or a5paper or ... 
%\geometry{landscape}                		% Activate for rotated page geometry
%\usepackage[parfill]{parskip}    		% Activate to begin paragraphs with an empty line rather than an indent
\usepackage{graphicx}				% Use pdf, png, jpg, or eps§ with pdflatex; use eps in DVI mode
								% TeX will automatically convert eps --> pdf in pdflatex		
\usepackage{amssymb}
\usepackage{amsmath}
\usepackage{amsthm}
\newtheorem{exercise}{Exercise}
%SetFonts

%SetFonts


\title{Beginning Exercises for the group on Radicals}
\author{}
%\date{today}							% Activate to display a given date or no date

\begin{document}
\maketitle
Here are some warmup exercises for the group on Radicals -- the goal is partly learning about radicals, partly getting used to using M2. If you're a beginner, get help from the experts (among them Justin and Ayah!) More experienced M2 users might want to start right away to make some benchmarks for testing radical and minimalPrimes strategies.

\begin{exercise}
The coefficients of the characteristic polynomial of an $n\times n$ matrix $M$ are in the radical of the ideal of entries of $minors(1, M^{n})$. Try generic matrices of small size
and make a conjecture about which powers of each are in this ideal. How about higher order minors of $M^{n}$ or of $M^{k}$ for other $k$? 
\end{exercise}

\begin{exercise}
In the situation of Huneke's Example 2.2, what power of $I_{n}(A)$ is in $(f,g)$. What is $(f,g): I_{n}(A)$?
\end{exercise}

\begin{exercise}
Write a ``while'' loop to implement the algorithm in Huneke's section 4; Put in a counter that will declare failure when Koll\'ar's bound is passed.
\end{exercise}
\maketitle
%\section{}
%\subsection{}



\end{document}  