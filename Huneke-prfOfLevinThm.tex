


\documentclass[11pt, oneside]{article}   	% use "amsart" instead of "article" for AMSLaTeX format
\usepackage{geometry}                		% See geometry.pdf to learn the layout options. There are lots.
\geometry{letterpaper}                   		% ... or a4paper or a5paper or ... 
%\geometry{landscape}                		% Activate for rotated page geometry
%\usepackage[parfill]{parskip}    		% Activate to begin paragraphs with an empty line rather than an indent
\usepackage{graphicx}				% Use pdf, png, jpg, or eps§ with pdflatex; use eps in DVI mode
	\usepackage{amsmath}
	\usepackage{amsthm}							% TeX will automatically convert eps --> pdf in pdflatex		
\usepackage{amssymb}
\newtheorem{Theorem}{Theorem}

%SetFonts

%SetFonts


\title{A Proof of Levin's Theorem}
\author{Craig Huneke}
%\date{}							% Activate to display a given date or no date

\begin{document}
\maketitle
%\section{}
%\subsection{}

These are informal notes by Craig Huneke, written while he was teaching a course from
the Queens Lecture notes by Gulliksen and Levin. The theorem below is stated by
Vasconcelos in his book ``Arithmetic of Blowup Algebras'', Thm 10.3.16, and attributed to 
Levin-personal communication. Vasconcelos gives a proof of the easy half,
but it seems that no proof of the harder half was published.

	--DE

\begin{Theorem} 
 Let $R$ be a local ring, $I$ an ideal in $R$ of finite projective
dimension. Let
$J$ be the ``content" of $I$ ,i.e. the ideal of all coefficients of relations
on a minimal set of generators of I. If r= $\mu(I) > pd_RR/I $ then
$(I:J)^r \subset I$.
\end{Theorem}

The proof depends on a lemma of Gulliksen's(Lemma 1.3.2 of Gulliksen-Levin)
on the extension of derivations.
If $A$ is a differential graded $R$-algebra,  a derivation $j$ of $A$ is an $R$-
linear mapping  $A \to A$ of degree $w$ such that for $x \in A_p$ , $y \in
A_q$
$$ j(xy)=(-1)^{wq}j(x)y + xj(y) $$

    Suppose $z$ is a cycle in $A_p$ and $B = \{ A\langle S \rangle ; dS =
z\}$, the algebra obtained by adjoining to $A$ a variable to kill $z$.
Gulliksen's lemma says that a derivation $j$ of $A$ of negative degree $-w$ may
be
extended   to $B$ if
and only if $j(d(A_{p+1})) \subset d(A_{p-w+1}) $

  In particular, if $A$ is an acyclic closure of the homomorphism $R \to R/I$,
(begin with the Koszul complex over $R$ on a set of generators of $I$ and
adjoin a sequence of variables $S_1,S_2,\ldots $ to kill all homology in
degrees $>$ 0), then an $R$-linear map $j:A_p \to R$ may be extended to derivation
of $A$ of degree $-p$ if and only if $j$ takes the $p$-boundaries to 0-
boundaries, i.e. elements of $I$, but this is precisely the condition that the
composite $A_p \to R \to R/I $ be a $p$-cocycle in $Hom_R(A,R/I)$. 

 
This leads to the following theorem
about the action of $Ext_R(R/I,R/I)$
on $Tor^R(R/I,R/I)$.

\begin{Theorem}
 Let $x_1,\ldots,x_s$ be elements in $Ext^1_R(R/I,R/I)$
 and let $y_1,\ldots,y_s$ be
elements of $Tor^R_1(R/I,R/I) $. Then
$$(x_1\ldots x_s)(y_1\ldots y_s) =  \det(M)$$
where $M$ is the $s \times s $ matrix
whose $i,j$-th entry is $x_iy_j$
\end{Theorem}
Here the product of the $x_i $ is the
Yoneda product while the product of the
$y_i$ is the usual
algebra product in $Tor^R(R/I,R/I)$).

\begin{proof}
The usual way to compute the action of    $Ext^1_R(R/I,R/I)$
 on $Tor^R(R/I,R/I) $ is to take a representative cocyle in
$Hom_R(A_1,R/I)$ , lift it to a map $\xi$ of degree -1 from $A \to A$
commuting with the differential and then apply $\xi \otimes 1$ to a cycle
in $A \otimes_R (R/I)$ representing a class in $Tor^R(R/I,R/I) $.
However, in this situation, we can take $\xi $ to be a derivation so
multiplication by any $x_i$ satisfies the derivation rule. The result follows. 
\end{proof}

\begin{Theorem}
 Let $I=(t_1,\ldots,t_r)$ and
let $f_1,\ldots,f_r$ be any $r$ 1-cocycles in $Hom_R(A,R/I) $, and
$T_1,\ldots,T_r$ a basis for $A_1$ with $dT_i = t_i$. (This
means that for any relation $\sum_{i=1}^ra_it_i =0 \in R$,  it is also true that
$\sum_{i=1}^ra_if_j(T_i) = 0 \in R/I $) Let $b_{ij} \in R $ represent the
elements $f_i(T_j) \in R/I$ and let $M$ be the $r \times r$ matrix with these
entries. Then all subdeterminants of M of order $> pd_RR/I$ are elements of
$I$.
\end{Theorem}

\begin{proof}
 Since each $dT_i \in I$, each $T_i \otimes 1 $ is a 1-cycle in
$A \otimes (R/I) $. Apply the theorem above.
\end{proof}
 
\begin{proof}[Proof of main result.]
If $b_1,\ldots,b_r  \in (I:J) $ the maps
%\begin{cases}
%  n/2  & n \text{ is even} \\
%  3n+1 & n \text{ is odd}
%\end{cases}

$$ 
f_i(T_j) = 
\begin{cases}
  b_i & j=i \\ 
  0 & j\neq i
\end{cases}
$$
when composed with the canonical map $R \to R/I$
 become  1-cocycles in $Hom_R(A,R/I) $.
The matrix $M$
is then diagonal and the determinant is $ b_1\ldots b_r \in I$
since
$Tor^R_r (R/I,R/I) = 0$. 
\end{proof}


\end{document}  


\bye

